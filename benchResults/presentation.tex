\documentclass{beamer}

\usepackage[cp1252,utf8]{inputenc}
\usepackage[dutch]{babel}
\usepackage{graphicx}
\usepackage{float}
\usepackage{amsmath}
\usepackage{geometry}
%\usepackage{mathtools}
\usepackage{multicol}
\usepackage{amsthm}
\usepackage{amssymb}
\usepackage{algorithm}
\usepackage{geometry}
\usepackage[noend]{algpseudocode}
\usepackage{pgfpages}
\usepackage{listings}
\usepackage{color}


\definecolor{gray}{rgb}{0.5,0.5,0.5}

\lstset{
	frame=tb,
	language=python,
	aboveskip=3mm,
	belowskip=3mm,
	showstringspaces=false,
	columns=flexible,
	basicstyle={\small\ttfamily},
	numbers=none,
	numberstyle=\tiny\color{gray},
	keywordstyle=\color{blue},
	commentstyle=\color{red},
	breaklines=true,
	breakatwhitespace=true,
	tabsize=3
}


\newcommand{\btVFill}{\vskip0pt plus 1filll}

%Information to be included in the title page:
\title{Project Advanced Programming}
\subtitle{N-body Simulation}
\author{Thomas Van Bogaert}
\date{}

\AtBeginSection[]
{
	\begin{frame}
		\frametitle{}
		\tableofcontents[currentsection]
	\end{frame}
}
\setbeamertemplate{footline}[frame number]
\setbeamertemplate{navigation symbols}{}

%\setbeameroption{show notes}
%\setbeameroption{show notes on second screen=right}

\newcommand{\light}[1]{\textcolor{gray}{#1}}

\begin{document}
	
	\frame{\titlepage}
	
	\begin{frame}{Basis van N-body simulatie}
		\note{}
		\begin{itemize}
			\item N lichamen met elk een positie en massa
			\item Doel: bereken de evolutie van elk lichaam over tijd
			\item Moeilijkheid: elk lichaam kan elk ander lichaam be\"invloeden
		\end{itemize}
	\end{frame}
	\begin{frame}{Berekenen van de evolutie van elk lichaam}
		\note{}
		\begin{itemize}
			\item Wiskundig voor te stellen door differentiaalvergelijking
			\item Benader de oplossing met numerieke integratie
		\end{itemize}
	\end{frame}
	\begin{frame}{Velocity Verlet integratie}
		\begin{itemize}
			\item Tweede orde benadering
			\item Naast positie nu ook snelheid en acceleratie nodig
			\item Als $\Delta t$ kleiner $\Rightarrow$ benadering preciezer
		\end{itemize}
		Op tijd $t$ worden de nieuwe positie en snelheid gegeven door:
		\begin{align*}
		\vec{x}(t + \Delta t) =& \vec{x}(t) + \vec{v}(t) \Delta t + \frac{1}{2} \vec{a}(t) \Delta t^2  \\
		\vec{v}(t + \Delta t) =& \vec{v}(t) + \frac{\vec{a}(t) + \vec{a}(t + \Delta t)}{2} \Delta t
		\end{align*}
	\end{frame}
	\begin{frame}{Velocity Verlet integratie}
		Probleem: wat als $\vec{a}(t_0)$ niet gegeven?
		\begin{align*}
		\vec{x}(t + \Delta t) =& 
		\vec{x}(t) + \vec{v}(t) \Delta t + \frac{1}{2} \vec{a}(t) \Delta t^2 \\
		\vec{v}(t + \Delta t) =& \begin{cases}
		\vec{v}(t) + \vec{a}(t + \Delta t) \Delta t    &\text{als } t = t_0 \\
		\vec{v}(t) + \frac{\vec{a}(t) + \vec{a}(t + \Delta t)}{2} \Delta t     &\text{als } t \neq t_0 \\
		\end{cases}
		\end{align*}
	\end{frame}
		
	\begin{frame}{Berekenen van $\vec{a}(t)$}
		\begin{itemize}
			\item In het algemeen: bereken de kracht tussen elk paar objecten
			\item Brute force: $O(n^2)$
			\item Barnes-hut benadering: $O(n\log(n))$
		\end{itemize}
	\end{frame}
	
	\defverbatim[colored]\codeBF{
		\begin{lstlisting}
		for bodyA in bodies:
			newAcceleration = (0, 0, 0)
			for bodyB in bodies:
				forceAB = force(bodyA, bodyB)
				newAcceleration += forceAB / massA
			updateVelocity(bodyA, newAcceleration)
			updateAcceleration(bodyA, newAcceleration)
		\end{lstlisting}
	}
	\begin{frame}{Brute force}
		\codeBF
		\begin{itemize}
			\item \texttt{forceAB} is een pure functie
			\item Zeer goed paralleliseerbaar
		\end{itemize}
	\end{frame}
	
	
	\begin{frame}{Implementatie in \texttt{C++}}
		In de implementatie berekent \texttt{forceAB} de aantrekkingskracht onder invloed van zwaartekracht.
		\\
		Brute force N-body simulatie implementaties:
		\begin{itemize}
			\item Single threaded (cpu-single-thread)
			\item Multi threaded met OpenMP (cpu-multi-thread)
			\item Na\"ive OpenCL implementatie (opencl)
			\item OpenCL implementatie gebruik makend van lokaal geheugen (openclloc)
			\item OpenCL implementatie met Struct of Arrays (SoA) i.p.v. Array of Structs (AoS) implementatie (openclvec)
		\end{itemize}
		Iedere implementatie is beschikbaar met double of single floating point precisie.
	\end{frame}
	
	\begin{frame}{Vergelijking van performantie}
		\begin{itemize}
			\item Nu komt de vergelijking van alle brute force implementaties
			\item Alle assen hebben logaritmische schaal ($\log(2)$)
			\item Gebruikte CPU: Xeon E3 1650 @3.6 Ghz
			\item Gebruikte GPU (OpenCL): AMD RX580 @1200 Mhz
			\item Linux kernel 4.20.3 (Antergos default scheduler)
			\item cpu-multi-thread implementatie is gebenchmarked met 6 cores op een 6 core machine
		\end{itemize}
	\end{frame}
	
	
	\begin{frame}{Comparison of CPU implementations}
		\begin{center}
			\resizebox{!}{.7\paperheight}{\input{compareCpu.tex}}
		\end{center}
	\end{frame}

	\begin{frame}{Compare different implementations (double)}
		\begin{center}
			\resizebox{!}{.7\paperheight}{\input{compareDifferentDouble.tex}}
		\end{center}
	\end{frame}	
	
	\begin{frame}{Speedup between different implementations (double)}
		\begin{center}
			\resizebox{!}{.7\paperheight}{\input{speedupDifferentDouble.tex}}
		\end{center}
	\end{frame}	
	
	\begin{frame}{Comparison of OpenCL implementations}
		\begin{center}
			\resizebox{!}{.7\paperheight}{\input{compareOpenCL.tex}}
		\end{center}
	\end{frame}

	\begin{frame}{Speedup between OpenCL implementations}
		\begin{center}
			\resizebox{!}{.7\paperheight}{\input{speedupOpenCL.tex}}
		\end{center}
	\end{frame}
	




%	\begin{frame}{Resultaten Brute Force Multithread: Speedup}
%		\begin{center}
%			\resizebox{!}{.7\paperheight}{\input{BFMTSpeedup.tex}}
%		\end{center}
%	\end{frame}
%
%	
%	\begin{frame}{Resultaten Brute Force Multithread: Strong scaling}
%		\begin{center}
%			\resizebox{!}{.7\paperheight}{\input{BFStrong.tex}}
%		\end{center}
%	\end{frame}
%	
%	\begin{frame}{Resultaten Brute Force GPU: Speedup}
%		\begin{center}
%			\resizebox{!}{.7\paperheight}{\input{BFOSpeedup.tex}}
%		\end{center}
%	\end{frame}

%	\defverbatim[colored]\codeBH{
%		\begin{lstlisting}
%		for body in bodies:
%			# traverses tree
%			newAcceleration = calculateAcceleration(body, root)  
%			updateVelocity(bodyA, newAcceleration)
%			updateAcceleration(bodyA, newAcceleration)
%		\end{lstlisting}
%	}
%
%	\begin{frame}{Barnes-Hut}
%		\note{
%			
%		}
%		\begin{itemize}
%			\item $O(n log(n))$
%			\item Deelt lichamen onder in octree
%			\item Het is niet nodig om krachten tussen elk paar lichamen te berekenen.
%			
%			 Schat verafgelegen lichamen af door het massacentrum
%			
%			\item Twee implementaties
%			\begin{itemize}
%				\item Iteratief gebruikmakend van space filling curve
%				\item Recursief
%			\end{itemize}
%		\end{itemize}
%		\codeBH
%	\end{frame}
	
%	\begin{frame}{Barnes-Hut algoritme}
%		\begin{center}
%			\includegraphics[width= 0.8\linewidth]{quadTreeBH.png}
%		\end{center}
%	\end{frame}
%
%	\begin{frame}{Resultaten Barnes-Hut Multithread: Speedup}
%		\begin{center}
%			\resizebox{!}{.7\paperheight}{\input{BHMTSpeedup.tex}}
%		\end{center}
%	\end{frame}
%
%	\begin{frame}{Resultaten Barnes-Hut Multithread: Strong scaling}
%		\begin{center}
%			\resizebox{!}{.7\paperheight}{\input{BHStrong.tex}}
%		\end{center}
%	\end{frame}
%
%	\begin{frame}{Resultaten Barnes-Hut GPU: Speedup (slowdown)}
%		\begin{center}
%			\resizebox{!}{.7\paperheight}{\input{BHOSpeedup.tex}}
%		\end{center}
%	\end{frame}
%
%	\begin{frame}{Resultaten Barnes-Hut Recursief}
%		\begin{center}
%			\resizebox{!}{.7\paperheight}{\input{BHTSpeedup.tex}}
%		\end{center}
%	\end{frame}
	
	\begin{frame}{Resultaat van simulatie}
		\begin{center}
%			\includegraphics[width=0.8\linewidth]{out.png}
%			\movie{\includegraphics[width=0.8\linewidth]{out.png}}{out2.ogv}
		\end{center}
	\end{frame}
	
%	\begin{frame}{Einde}
%		\begin{center}
%			Nog vragen?
%		\end{center}
%	\end{frame}
	
\end{document}

