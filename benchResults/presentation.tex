\documentclass{beamer}

\usepackage[cp1252,utf8]{inputenc}
\usepackage[dutch]{babel}
\usepackage{graphicx}
\usepackage{float}
\usepackage{amsmath}
\usepackage{geometry}
%\usepackage{mathtools}
\usepackage{multicol}
\usepackage{amsthm}
\usepackage{amssymb}
\usepackage{algorithm}
\usepackage{geometry}
\usepackage[noend]{algpseudocode}
\usepackage{pgfpages}
\usepackage{listings}
\usepackage{color}


\definecolor{gray}{rgb}{0.5,0.5,0.5}

\lstset{
	frame=tb,
	language=python,
	aboveskip=3mm,
	belowskip=3mm,
	showstringspaces=false,
	columns=flexible,
	basicstyle={\small\ttfamily},
	numbers=none,
	numberstyle=\tiny\color{gray},
	keywordstyle=\color{blue},
	commentstyle=\color{red},
	breaklines=true,
	breakatwhitespace=true,
	tabsize=3
}


\newcommand{\btVFill}{\vskip0pt plus 1filll}

%Information to be included in the title page:
\title{Project Advanced Programming}
\subtitle{N-body Simulation}
\author{Thomas Van Bogaert}
\date{}

\AtBeginSection[]
{
	\begin{frame}
		\frametitle{}
		\tableofcontents[currentsection]
	\end{frame}
}
\setbeamertemplate{footline}[frame number]
\setbeamertemplate{navigation symbols}{}

%\setbeameroption{show notes}
%\setbeameroption{show notes on second screen=right}

\newcommand{\light}[1]{\textcolor{gray}{#1}}

\begin{document}
	
	\frame{\titlepage}
	\begin{frame}{Wat is een N-body simulatie?}
		\note{
			
		}
		\begin{itemize}
			\item Een simulatie van de werking van krachten tussen verschillende objecten
			\item In het algemeen moet de kracht tussen elk paar objecten berekend worden
			\item Bevoorbeeld: interacties tussen geladen deeltjes of interactie van zwaartekracht tussen sterren
		\end{itemize}
		
	\end{frame}
	
%	\begin{frame}{Waarom een N-body simulatie?}
%		\begin{itemize}
%			\item Verschillende algoritmes beschikbaar (brute force, barnes-hut)
%			\item Goed voorbeeld van een paralleliseerbaar systeem
%		\end{itemize}
%	\end{frame}
	\begin{frame}
			Ik heb het geval van zwaartekracht verder uitgewerkt en twee algoritmes ge\"implementeerd:
			\begin{itemize}
				\item Brute force
				\item Barnes-Hut
			\end{itemize}
			In het implementeren heb ik mij vooral gefocused op performantie
	\end{frame}
	
	\begin{frame}{Uitleg resultaten}
		\note{Vooraleer we naar de resultaten van het brute force algoritme kijken}
		\begin{itemize}
			\item Speedup: Hoeveel keer sneller wordt het algoritme als er meerdere cores gebruikt worden
			\item \b Strong scaling: Voor hoeveel overhead zorgt het parellalisme? Geeft effici\"entie t.o.v. single thread
		\end{itemize}
	\end{frame}
	
	\defverbatim[colored]\codeBF{
		\begin{lstlisting}
		for bodyA in bodies:
			newAcceleration = (0, 0, 0)
			for bodyB in bodies:  # bodyB wordt nooit aangepast
				if bodyA != bodyB:
					forceAB = force(bodyA, bodyB)
					newAcceleration += forceAB / massA
			updateVelocity(bodyA, newAcceleration)
			updateAcceleration(bodyA, newAcceleration)
		\end{lstlisting}
	}
	
	\begin{frame}{Brute force}
		\begin{itemize}
			\item $O(n^2)$
			\item Zeer goed paralleliseerbaar.
		\end{itemize}
		\codeBF
	\end{frame}
	\begin{frame}{Speedup between OpenCL Implementations}
		\begin{center}
			\resizebox{!}{.7\paperheight}{\input{speedupOpenCL.tex}}
		\end{center}
	\end{frame}
	
	\begin{frame}{Speedup between different implementations (double)}
		\begin{center}
			\resizebox{!}{.7\paperheight}{\input{speedupDouble.tex}}
		\end{center}
	\end{frame}
%	\begin{frame}{Resultaten Brute Force Multithread: Speedup}
%		\begin{center}
%			\resizebox{!}{.7\paperheight}{\input{BFMTSpeedup.tex}}
%		\end{center}
%	\end{frame}
%
%	
%	\begin{frame}{Resultaten Brute Force Multithread: Strong scaling}
%		\begin{center}
%			\resizebox{!}{.7\paperheight}{\input{BFStrong.tex}}
%		\end{center}
%	\end{frame}
%	
%	\begin{frame}{Resultaten Brute Force GPU: Speedup}
%		\begin{center}
%			\resizebox{!}{.7\paperheight}{\input{BFOSpeedup.tex}}
%		\end{center}
%	\end{frame}

	\defverbatim[colored]\codeBH{
		\begin{lstlisting}
		for body in bodies:
			# traverses tree
			newAcceleration = calculateAcceleration(body, root)  
			updateVelocity(bodyA, newAcceleration)
			updateAcceleration(bodyA, newAcceleration)
		\end{lstlisting}
	}

	\begin{frame}{Barnes-Hut}
		\note{
			
		}
		\begin{itemize}
			\item $O(n log(n))$
			\item Deelt lichamen onder in octree
			\item Het is niet nodig om krachten tussen elk paar lichamen te berekenen.
			
			 Schat verafgelegen lichamen af door het massacentrum
			
			\item Twee implementaties
			\begin{itemize}
				\item Iteratief gebruikmakend van space filling curve
				\item Recursief
			\end{itemize}
		\end{itemize}
		\codeBH
	\end{frame}
	
%	\begin{frame}{Barnes-Hut algoritme}
%		\begin{center}
%			\includegraphics[width= 0.8\linewidth]{quadTreeBH.png}
%		\end{center}
%	\end{frame}
%
%	\begin{frame}{Resultaten Barnes-Hut Multithread: Speedup}
%		\begin{center}
%			\resizebox{!}{.7\paperheight}{\input{BHMTSpeedup.tex}}
%		\end{center}
%	\end{frame}
%
%	\begin{frame}{Resultaten Barnes-Hut Multithread: Strong scaling}
%		\begin{center}
%			\resizebox{!}{.7\paperheight}{\input{BHStrong.tex}}
%		\end{center}
%	\end{frame}
%
%	\begin{frame}{Resultaten Barnes-Hut GPU: Speedup (slowdown)}
%		\begin{center}
%			\resizebox{!}{.7\paperheight}{\input{BHOSpeedup.tex}}
%		\end{center}
%	\end{frame}
%
%	\begin{frame}{Resultaten Barnes-Hut Recursief}
%		\begin{center}
%			\resizebox{!}{.7\paperheight}{\input{BHTSpeedup.tex}}
%		\end{center}
%	\end{frame}
	
	\begin{frame}{Resultaat van simulatie}
		\begin{center}
%			\includegraphics[width=0.8\linewidth]{out.png}
%			\movie{\includegraphics[width=0.8\linewidth]{out.png}}{out2.ogv}
		\end{center}
	\end{frame}
	
	\begin{frame}{Einde}
		\begin{center}
			Nog vragen?
		\end{center}
	\end{frame}
	
\end{document}

